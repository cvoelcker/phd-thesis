\chapter{Auxiliary Lemmata}
\label{app:formal}

\section{\autoref{chap:understanding}: Helpful definitions and lemmata}

\label{app:formal:understanding}

This section contains helpful lemmata that are used for our proofs.
Where we took these from existing work, we provide references, otherwise the proofs are our own, although probably also known in the literature.

\subsection{Linear Algebra}
\label{app:formal:understanding:linalg}

As before, for matrices $V$ we write $\mathrm{span}(V)$ to denote the span of their column vectors. 

\begin{definition}[Top-$k$ singular vectors]\label{def:understanding:topk-single}
Let $P^\pi = U^\top \Sigma V$ be the singular value decomposition, and assume that the diagonal of $\Sigma$ is arranged in decreasing order.
The first $k$ rows of $U^\top$ and the first $k$ columns of $V$ are called the top-k left and right singular vectors, respectively.
\end{definition}

\begin{lemma}[Spectrum of Kronecker product matrix]\label{lem:understanding:spectrum_kronecker}
    Consider two non-singular matrices $M$ and $N$. Let $\lambda_i$ be the eigenvalues of $M$ and $\mu_j$ be the eigenvalues of $N$, with eigenvectors $u_i$ and $v_j$ respectively. Then the eigenvalues of $M\otimes N$ are $\lambda_i\mu_j$ with eigenvectors $u_i \otimes v_j$ respectively.
\end{lemma}

\begin{proof}
For any eigenvector $u_i$ of $M$ and $v_j$ of $N$, we have
    $$\left(M\otimes N\right) \left(u_i \otimes v_j\right) = \left(M u_i\right) \otimes \left(N v_j\right) = \left(\lambda_i u_i\right)\otimes\left(\mu_j v_j\right) = \lambda_i\mu_j \left(u_i \otimes v_j\right).$$

    So $u_i\otimes v_j$ is an eigenvector of $M\otimes N$ with eigenvalue $\lambda_i \mu_j$
\end{proof}

\begin{lemma}[Orthogonalized bases and the Kronecker product]\label{lem:understanding:orth_kronecker}
    Let $V \in \vecR{n}{m}$ be a matrix. Let $\mathrm{orth}(V)$ be a matrix of any orthonormal basis vectors for the column vectors of $V$. Then $\mathrm{span}(V\otimes \vone) = \mathrm{span}(\mathrm{orth}(V) \otimes \vone)$. Furthermore $(1/k) \mathrm{orth}(V) \otimes \vone$ is an orthonormal basis of $\mathrm{span}(V \otimes \vone)$.
\end{lemma}

\begin{proof}
    Let $V_i \otimes \vone^k$ be any column vector of $V\otimes \vone^k$. Let $\alpha_1,\dots,\alpha_m$ be coefficients so that $V_i = \sum_{j=1}^m \alpha_j \mathrm{orth}(V)_j$. Then $$\sum_{j=1}^m \alpha_j \left(\mathrm{orth}(V)_j \otimes \vone^k\right) = \left(\sum_{j=1}^m \alpha_j \mathrm{orth}(V)_j\right) \otimes \vone^k = V_i \otimes \vone^k,$$ following the standard associative and distributive properties of the Kronecker product.

    As every vector in the original span can be represented in the orthogonalized span, the two are equivalent.

    Finally note that $$\frac{1}{k}\left(\mathrm{orth}(V) \otimes \vone\right)_i^\top \left(\mathrm{orth}(V) \otimes \vone\right)_i = \frac{1}{k}\sum_{j=1}^k \left(\mathrm{orth}(V)_i^\top \mathrm{orth}(V)_i^\top\right) = 1,\,\text{and}$$ $$\frac{1}{k}\left(\mathrm{orth}(V) \otimes \vone\right)_j^\top \left(\mathrm{orth}(V) \otimes \vone\right)_i = \frac{1}{k}\sum_{t=1}^k \left(\mathrm{orth}(V)_j^\top \mathrm{orth}(V)_i^\top\right) = 0.$$
\end{proof}

\begin{lemma}[Reduced rank regression]\label{lem:understanding:rrr}
    Let $C,D \in \vecR{n}{n}$, $A\in\vecR{n}{k}$, and $B\in\vecR{k}{n}$ be full rank matrices with $n \geq k$. Let $A^*, B^* = \min_{A,B} \lVert CAB - DC\rVert^2_F$. Let ${u_1,\dots,u_k}$ and ${v_1\dots,v_k}$ be the top-k singular vectors of $C^{-1}DC$ according to \autoref{def:understanding:topk-single}. Then $\mathrm{span}(A^*) = \mathrm{span}({u_1,\dots,u_k})$ and $\mathrm{span}(B) = \mathrm{span}({v_1,\dots,v_k})$.
\end{lemma}

This is a reduced rank regression problem \parencite{izenman1975reduced} or low-rank matrix approximation problem. 

The unconstrained solution to the problem is given by $\hat{A}\hat{B} = C^{-1}DC$. From the Eckhart-Young theorem, we know that the optimal low-rank approximation to $C^{-1}DC$ is given by the top-k singular vectors.
Therefore $A$ and $B$ span top-$k$ left and right singular vectors respectively.
For a more extensive proof, please refer to \textcite{izenman1975reduced}.

\subsection{Stochastic matrices}
\begin{lemma}[Spectrum of a resolvent matrix]\label{lem:understanding:spectrum_resolvent}
    Let $A$ be an invertible matrix with unique real eigenvalues and $-1 \leq \lambda_{\min} \leq \lambda_{\max} \leq 1 $. Let $\gamma\in(0,1)$.
    The matrices $A$ and $(I - \gamma A)^{-1}$ have the same eigenvectors and the ordering of the corresponding eigenvalues remains the same.
\end{lemma}

\begin{proof}
    Let $e$ be an eigenvector of $A$ and $\lambda$ the corresponding eigenvalue. Then $$(I - \gamma A)^{-1}e = \sum_{i=0}^n \gamma^n A^ne = \sum_{i=0}^n (\gamma\lambda)^n e = \frac{1}{1 - \gamma\lambda}e.$$ As $(1-\gamma\lambda)^{-1}$ is a monotonic function for $-1 \leq \lambda \leq 1$ and $\gamma\in[0,1]$, the ordering of the eigenvalues remains the same.
\end{proof}

\begin{lemma}[Basis equivalence of linear reward and value function]\label{lem:understanding:spectrum_rew_value}
    Let $r$ be the vector representation of the reward function and $V$ of the value function respectively for an MDP with fixed policy and transition matrix $P^\pi$.
    Let $U = \{u_1,\dots,u_n\}$ be the set of eigenvectors of $P^\pi$, and let $U^r \subseteq U$ be a minimal set of eigenvectors so that $r \in \mathrm{span}(U^r)$.
    Then $V \in \mathrm{span}(U^r)$.
\end{lemma}
\begin{proof}
    Let $r = \sum_{i=1}^k \alpha_i\, u^r_i$ be the reward representation in the basis $U^r$. Then, by \autoref{lem:understanding:spectrum_resolvent} $$V^\pi = (I - \gamma P^\pi)^{-1} r = \sum_{i=1}^k \frac{\alpha}{1 - \lambda_i\gamma}  u^r_i.$$
\end{proof}


\subsection{Ordinary differential equations}

For describing the stability of ODEs, we use the notion of \emph{asymptotic stability}, with the condition $\Re(\lambda_i) < 0$ for all eigenvalues $\lambda_i$ of the Jacobian at a critical point.

\begin{lemma}[Linear reparameterization of an autonomous ODE]
\label{lem:understanding:stability}
Let $y' = f(y)$ be an autonomous ordinary differential equation. Let $y^*$ be any critical point for which $f(y^*) = 0$. Let furthermore $x' = f(A x)$ be a reparameterized autonomous ODE for any invertible matrix $A$.
Then the $x^*$ is a critical point with $f(Ax) = 0$ iff $x^* = A^{-1}y^*$.
Furthermore, the eigenvalues of the Jacobian of $y'$ at $y^*$ are equal to the eigenvalues of the Jacobian of $x'$ at $x^* = A^{-1}y^*$.
\end{lemma}

\begin{proof}
The direction $x = A^{-1}y^* \Rightarrow f(Ax) = 0$ is clear by direct evaluation. We now focus on the direction $f(Ax) = 0 \Rightarrow x = A^{-1}y^*$. Assume that $f(Ax)$ is $0$ and $x = A^{-1}y$ for a $y$ which is not a point satisfying $f(y)=0$. But then $0 = f(AA^{-1}y) = f(y) \neq 0$, a contradiction.

For stability, note that
\begin{alignat*}{2}
    &\qquad&\ddt{t} y & = f(y) \\
    \implies&&\ddt{t} x & = \ddt{y} x \ddt{t} y\\
    && &= A^{-\top} f(y)\\
    \implies&& \ddt{x} \ddt{t} x &= A^{-\top} \ddt{x} f(A x)\\
    && &= A^{-\top}  \ddt{y} f(y) \ddt{x} A x\\
    && &= A^{-\top}  \ddt{y} f(y) A^\top.
\end{alignat*}
This shows that the Jacobian $\ddt{x} f(Ax)|_{x=A^{-1}y_0}$ is similar to the Jacobian $\ddt{y} f(y)|_{y=y_0}$, which means their spectra are identical.
\end{proof}


\subsection{MDP representation and TD learning}

\begin{lemma}[Lemma 5 of \textcite{tang2022understanding}]\label{prop:understanding:TangResult2}
    Suppose $P^\pi$ is real-diagonalizable, and write $u_1,\dots, u_n$ for its eigenvectors. Then any orthonormal matrix $\Phi$ which has the same span as a set of $k$ eigenvectors is a minimizer of $L_{\text{lat}}$.
\end{lemma}


The next three statements are taken from \textcite{ghosh2020representations}.  They address the TD loss wrt to the learned weights $\hat{V}$ and fixed $\Phi$ (compare \autoref{sec:understanding:background}).
We changed the notation of the statements to fit our notation here, we have denoted the diagonal matrix of the state distribution as $D$ and assume that $D=I$ in \autoref{assumption:understanding:1}, while \textcite{ghosh2020representations} uses $\Xi$. They use $\theta$ for the value function weights while we use $\hat{V}$.
They also uses $\text{Spec}(A)$ to denote the spectrum, the set of all eigenvalues of a matrix $A$.

The notion of stability used in \textcite{ghosh2020representations} is that of convergence to the unique fixed point of the projected Bellman update of the linear ODE induced by the $L_\text{TD}$ loss when fixing $\Phi$. In the two-timescale scenario considered in this paper, this corresponds to the ``inner'' ODE over $\hat{V}$.

\begin{lemma}[Proposition 3.1 of \textcite{ghosh2020representations}]\label{prop:understanding:gosh1}
    TD(0) is stable if and only if the eigenvalues of the implied iteration matrix $A_\Phi = \Phi^\top D (I - \gamma P^\pi) \Phi$ have positive real components, that is $$\text{Spec}\left(A_\Phi\right) \subset \mathbb{C}_+ \coloneq \{z : \text{Re}(z) > 0 \}.$$
    We say that a particular choice of representation $\Phi$ is stable for $(P^\pi, \gamma, D)$ when $A_\Phi$ satisfies the above condition.
\end{lemma}

\begin{lemma}[Proposition 3.2 of \textcite{ghosh2020representations}]\label{prop:understanding:gosh2}
    An orthogonal representation $\Phi$ is stable if and only if the real part of the eigenvalues of the induced transition matrix $\Pi P^\pi \Pi$ where $\Pi = \Phi\Phi^\top$ is bounded above, according to $$\text{Spec}\left(\Pi P^\pi \Pi\right) \subset \{z \in \mathbb{C}: \text{Re}(z) < \frac{1}{\gamma} \}.$$
    In particular, $\Phi$ is stable if $\rho(\Pi P^\pi\Pi) < \frac{1}{\gamma}$.
\end{lemma}

\begin{lemma}[Theorem 4.1 of \textcite{ghosh2020representations}]\label{prop:understanding:gosh3}
    An orthogonal invariant representation $\Phi$ (meaning $\text{span}(P^\pi \Phi) \subseteq \text{span}(\Phi)$) satisfies $$\text{Spec}\left(\Pi P^\pi \Pi\right) \subseteq \text{Spec}(P^\pi) \cup \{0\}$$ and is therefore stable.
\end{lemma}

As a corollary of their proof we have that 

\begin{lemma}[Corollary of \textcite{ghosh2020representations}]\label{lem:understanding:gosh:corr}
    Let $\Phi$ be an orthogonal (but not necessarily square) invariant representation of $P^\pi$. Then the spectral radius $\rho(\Phi^\top P^\pi \Phi) \leq 1.$ 
\end{lemma}

The proof follows directly from \autoref{prop:understanding:gosh3} by the cyclical property of the spectrum.
    

The following two results are our own, although closely related results exist in the literature.

\begin{lemma}[Lossless approximation of $V$]\label{lem:understanding:lossless_approx}
    Let $\Phi$ be an orthonormal basis of an invariant subspace of $P^\pi$ and let \autoref{assumption:understanding:low_rank} hold. Let $V = (I - \gamma P^\pi)^{-1} r$ be the value function of $P^\pi$ and $r^\pi$. Then $$\Phi (I - \gamma \Phi^\top P^\pi \Phi)^{-1} \Phi^\top r^\pi = V.$$
\end{lemma}
\begin{proof}
    By \autoref{assumption:understanding:low_rank} we can find a matrix $A$ so that $r^\pi = \Phi A$ and $\Phi\Phi^\top r^\pi = r^\pi$, and by the invariant subspace assumption, we can find a matrix $B$ so that $P^\pi \Phi = \Phi B$.
    Writing the inverted matrix as an infinite sum, which is valid as the spectrum of $\Phi^\top P^\pi \Phi$ is bounded by 1 following from \autoref{lem:understanding:gosh:corr} and Carl Neumann's theorem over power series, we obtain
    \begin{align*}
        \Phi (I - \gamma \Phi^\top P^\pi \Phi)^{-1} \Phi r^\pi &= \Phi \sum_{n=0}^\infty \gamma^n (\Phi^\top P^\pi \Phi)^n \Phi r^\pi \\
        &= \Phi \sum_{n=0}^\infty \gamma^n B^n \Phi^\top r^\pi &\quad (P^\pi \Phi = \Phi B)~\text{and}~(\Phi^\top \Phi = I)\\
        &= \sum_{n=0}^\infty \gamma^n \Phi B^n \Phi^\top r^\pi \\
        &= \sum_{n=0}^\infty \gamma^n {P^\pi}^n \Phi\Phi^\top r^\pi &\quad (\Phi B = P^\pi \Phi)~\text{iterated} \\
        &= \sum_{n=0}^\infty \gamma^n {P^\pi}^n r^\pi &\quad (\Phi\Phi^\top r^\pi = r^\pi) \\
        &= V
    \end{align*}
\end{proof}



\section{\autoref{chap:cvaml}: Helpful definitions and lemmata}
\label{app:formal:cvaml}

We provide helper lemmata for \autoref{sec:cvaml:theory_1} and \autoref{sec:cvaml:theory_2} in this section.

All of our proofs rely heavily on a standard expansion technique which is used to prove that sample-based losses correctly approximate population losses.
This is found in standard textbooks on learning theory such as \textcite{gyrfi2002adt}. 

It proceeds by expanding a loss $\EEX{x\sim p, y\sim q(\cdot|x)}{\left|f(x) - y\right|^2}$ with an expected target $f^*(x)$.
In our case, this is mostly the minimizer of the target loss when evaluating surrogate losses.
Then we obtain $$\EEX{x\sim p, y\sim q(\cdot|x)}{\left|f(x) - y\right|^2} = \EEX{x\sim p, y\sim q(\cdot|x)}{\left|f(x) - f^*(x) + f^*(x) y\right|^2},$$
and continue expanding from there.
Issues generally arise when $y$ depends on $f(x)$ in some way, or when $f(x)$ itself involves a sampling procedure.

\subsection{\autoref{sec:cvaml:theory_1} Lemmata}

For the following results, we will use the following notation.
Let $\mathcal{X}$ be a discrete sample space. 
Let $f: \mathcal{X} \rightarrow \mathbb{R}$ be a random variable.
Let $\P^*$ be the set of distributions for which both $\P^* = \Argmin_{p'} \mathrm{Var}_{p'}[f(x)]$, and for all $p^* \in \P^*$ $\EEX{p^*}{f(x)} = \EEX{p}{f(x)}$.
Let $k$ be an integer.
Finally, let $g(\xi) = \EEX{x \sim p}{\left(\EEX{y \sim \xi}{f(y)} - f(x)\right)^2} + \frac{1}{k}\mathrm{Var}_\xi[f(x)].$

We assume the following condition on a distribution $p$ and a function $f$ for all the lemmata in this section.
\begin{assumption}
Let $p$ be a probability distribution over $\mathcal{X}$ with $\EEX{p}{f(x)}$, for which no $x$ exists so that $f(x) = \EEX{p}{f(x)}$.
This is an assumption both on $f$ and $p$.
\end{assumption}
This is an important assumption as it guarantees that there is no distribution $q$ with $0$ variance such that $\EEX{q}{f(x)} = \EEX{p}{f(x)}.$
This excludes a corner case of our proof, as fully deterministic environments and models do not lead to an uncalibrated IterVAML loss.

We now obtain three simple lemmata about the minima of the function $g$.

\begin{lemma}\label{lemma:cvaml:3}
There does not exist a distribution $p'$ such that $p'\not\in \P^*$, $\EEX{p'}{f(x)} = \EEX{p}{f(x)}$, and $g(p') \leq g(p^*)$ for any $p^*\in\P$.
\end{lemma}

\begin{proof}
    To prove the lemma, we first evaluate $g$ for any distribution $\xi^*$ with $\EEX{\xi^*}{f(x)} = \EEX{p}{f(x)}$ 
    \begin{align}
        g(\xi^*) &= \EEX{x \sim p}{\left(\EEX{y \sim \xi^*}{f(y)} - f(x)\right)^2} + \frac{1}{k}\mathrm{Var}_\xi[f(x)]\\
        &= \EEX{x \sim p}{\left(\EEX{y \sim p}{f(y)} - f(x)\right)^2} + \frac{1}{k}\mathrm{Var}_{\xi^*}[f(x)]\\
        &= \mathrm{Var}_{p}(f(x)) + \frac{1}{k}\mathrm{Var}_{\xi^*}[f(x)]
    \end{align}
    
    As we constructed the set $\P^*$ so that all members have equal (minimum) variance, we obtain the same $g(p^*)$ for all $p^*\in\P^*$.

    We now set up a contradiction by assuming $p'$ exists. By this assumption
    \begin{align}
        g(p') &< g(p^*) \\
        \mathrm{Var}_{p}(f(x)) + \frac{1}{k}\mathrm{Var}_{p'}[f(x)] &< \mathrm{Var}_{p}(f(x)) + \frac{1}{k}\mathrm{Var}_{p^*}[f(x)]\\
        \mathrm{Var}_{p'}[f(x)] &< \mathrm{Var}_{p^*}[f(x)]
    \end{align}

    For the function $p'$ to exist so that $g(p') < g(p^*)$, we would therefore require $\mathrm{Var}_{p'}[f(x)] < \mathrm{Var}_{p^*}[f(x)]$, which is impossible by the definition of $\P^*$.
    This is a contradiction with the requirements for $p'$ in the lemma, which concludes our proof.
\end{proof}

In the following lemma, we show that under some conditions a distribution $q$ exists which has $g(q) < g(p^*)$ and for which $\EEX{p}{f(x)} \neq \EEX{q}{f(x)}$.
We can rewrite $g$ as
\begin{align}
    g(q) &=\EEX{p}{\left(\EEX{q}{f(y)} - f(x)\right)^2} + \frac{1}{k}\mathrm{Var}_q[f(x)] \\
    &= \EEX{p}{f(x)^2} - 2\EEX{p}{f(x)}\EEX{q}{f(x)} + \EEX{q}{f(x)}^2 + \frac{1}{k}\mathrm{Var}_q[f(x)]\\
    & \leq \EEX{p}{f(x)^2} - 2\EEX{p}{f(x)}\EEX{q}{f(x)} + \EEX{q}{f(x)^2}  + \frac{1}{k}\mathrm{Var}_q[f(x)]\label{eq:cvaml:jensen}\\
    & = \mathrm{Var}_p[f(x)] + \frac{k + 1}{k}\mathrm{Var}_q[f(x)] + (\EEX{p}{f(x)} - \EEX{q}{f(x)})^2
\end{align}
\autoref{eq:cvaml:jensen} follows from Jensen's inequality.
Intuitively, the function $g$ depends on the squared deviation of the expectation and the variance of both $p$ and $q$.
If the variance of $q$ can be reduced more than the squared deviation of the means, then $\P^*$ will not contain the minimum of $g$.

Note that the conditions on $q$ are sufficient but not necessary, as we use Jensen's inequality to obtain our bound.
In addition, to simplify the proof, we assume that $q$ is a distribution with zero variance.
In practice, any distribution with $\frac{k + 1}{k}\mathrm{Var}_q[f(x)] + (\EEX{p}{f(x)} - \EEX{q}{f(x)})^2 < \mathrm{Var}_{p^*}[f(x)]$ and $\EEX{q}{f(x)} \not = \EEX{p}{f(x)}$ will suffice, but this requirement is somewhat less intuitive to grasp.
As our definition of calibration does not require us to exhaustively characterize all cases for $p$, $p^*$, and $q$, we have chosen this set of conditions which simplifies the proof. 
As our Garnet experiments show, many randomly generated transition distributions admit distributions where the minimizer of $g$ does not match $p$ in expectation.

\begin{lemma}\label{lemma:cvaml:4}
    If there exists a distribution $q$ with $(\EEX{q}{f(x)} - \EEX{p}{f(x)})^2 < \frac{1}{k} \mathrm{Var}_{p^*}[f(x)]$, and $\mathrm{Var}_{q}[f(x)] = 0$, then $g(q) < g(p^*)$ for all $p^*$ and, by the assumptions on $p$ and $f$, $\EEX{x\sim q}{f(x)} \neq \EEX{x \sim p}{f(x)}$.
\end{lemma}
\begin{proof}
    Choose any $p^*\in\P^*$. As $\mathrm{Var}_q[f(x)] = 0$, $g(q) = \EEX{x\sim p}{\left(\EEX{y\sim q}{f(y)} - f(x)\right)}$. We can now decompose $g(q)$ as 
    \begin{align}
        g(q) = \EEX{p}{\left(\EEX{q}{f(y)} - f(x)\right)^2} &= \EEX{p}{f(x)^2} - 2\EEX{p}{f(x)}\EEX{q}{f(x)} + \EEX{q}{f(x)}^2 \\
        & = \mathrm{Var}_p[f(x)] + (\EEX{p}{f(x)} - \EEX{q}{f(x)})^2 \\
        &< \mathrm{Var}_p[f(x)] + \frac{1}{k}\mathrm{Var}_{p^*}[f(x)] = g(p^*) \label{eq:cvaml:final_line}
    \end{align}
    \autoref{eq:cvaml:final_line} follows from the assumption on q.
    
    By the assumptions on $p$ and $f$ there does not exist a $x\in\mathcal{X}$ so that $f(x) = \EEX{p}{f(x)}$. However, as $\mathrm{Var}_q[f(x)] = 0$, $\EEX{q}{f(x)} = f(x_i)$ for some $x_i\in\mathcal{X}$. Therefore, $\EEX{q}{f(x)} \neq \EEX{p}{f(x)}$, which concludes the proof.
\end{proof}

As a consequence, we obtain the following, final lemma.

\begin{lemma}\label{lemma:cvaml:5}
    Let $\P^\mathbb{E}$ be the set of all distributions $\xi$ so that $\EEX{p}{f(x)} = \EEX{\xi}{f(x)}.$
    Assume $q$ satisfying \autoref{lemma:cvaml:4} exists.

    Then $\argmin_\xi g(\xi) \not \subseteq \P^\mathbb{E}$.
\end{lemma}

\begin{proof}
    The lemma is a direct consequence of \autoref{lemma:cvaml:3} and \autoref{lemma:cvaml:4}. 
    By \autoref{lemma:cvaml:3}, for all $p' \in \P^\mathbb{E}$ $g(p') \geq g(p^*)$. As a consequence of \autoref{lemma:cvaml:4}, there exists at least one $q$ with $g(q) < g(p^*)$ and that $q$ has $\EEX{q}{f(x)} \neq \EEX{p}{f(x)}.$ Then the statement follows directly as $q \not \in \P^\mathbb{E}.$
\end{proof}


The following lemma shows that a function that minimizes a quadratic and a variance term cannot be the minimum function of the quadratic. This is used to show that the minimum of the MuZero value function learning term is not the same as applying the model-based Bellman operator.

\begin{lemma}
\label{lemma:cvaml:muzero}
Let $g: \mathcal{X} \rightarrow \mathbb{R}$ be a function that is not constant and let $\mu$ be a probability distribution over $\mathcal{X}$. 
Let $\mathcal{L}\roundb{f} = \mathbb{E}_{x\sim\mu}\squareb{\roundb{f(x) - g(x)}^2} + \mathbb{E}_{x\sim\mu}\squareb{f(x) g(x)} - \mathbb{E}_{x\sim\mu}\squareb{f(x)}\mathbb{E}_\mu\squareb{g(x)}$.
There exists a function space $\mathcal{F}$ with $g \in \mathcal{F}$ so that $g \notin \argmin_{f \in \mathcal{F}} \mathcal{L}(f)$.
\end{lemma}

\begin{proof}
The proof follows by showing that there is a descent direction from $g$ that improves upon $\mathcal{L}$. For this, we construct the auxiliary function $\hat{g}(x) = g(x) - \varepsilon g(x)$.
Evaluating $\mathcal{L}(\hat{g})$  yields 
\begin{align*}
     &~\varepsilon^2 \mathbb{E}_\mu\squareb{g(x)^2} + \mathbb{E}_\mu\squareb{\roundb{g(x) - \varepsilon g(x)} g(x)} \\
     &~- \mathbb{E}_\mu\squareb{\roundb{g(x) - \varepsilon g(x)}}\mathbb{E}_\mu\squareb{g(x)}\\
     =&~\varepsilon^2 \mathbb{E}_\mu\squareb{g(x)^2} + \roundb{1-\varepsilon}\mathbb{E}_\mu\squareb{g(x)^2} - \roundb{1 - \varepsilon} \mathbb{E}_\mu\squareb{g(x)}^2.
\end{align*}

Taking the derivative of this function wrt to $\varepsilon$ yields
\begin{align*}
     &~\frac{\mathrm{d}}{\mathrm{d} \varepsilon}\varepsilon^2 \mathbb{E}_\mu\squareb{g(x)^2} + \roundb{1-\varepsilon}\mathbb{E}_\mu\squareb{g(x)^2} - \roundb{1 - \varepsilon} \mathbb{E}_\mu\squareb{g(x)}^2\\
     =&~2\varepsilon~\mathbb{E}_\mu\squareb{g(x)^2} -\mathbb{E}_\mu\squareb{g(x)^2} + \mathbb{E}_\mu\squareb{g(x)}^2.
\end{align*}

Setting $\varepsilon$ to 0, we obtain
\begin{align*}
     &~\mathbb{E}_\mu\squareb{g(x)}^2 - \mathbb{E}_\mu\squareb{g(x)^2} = \text{Var}_\mu\squareb{g(x)}
\end{align*}

By the Cauchy-Schwartz inequality, the variance is only $0$ for a $g(x)$ constant almost everywhere. However, this violates the assumption.
Therefore there exists an $\varepsilon > 0$ so that $\mathcal{L}\roundb{\hat{g}} \leq \mathcal{L}\roundb{g}$.
We now can construct the function space $\mathcal{F}$ so that it includes at least $g$ and $g + \varepsilon g.$

\end{proof}

\subsection{Propositions from Bertsekas \& Shreve}
\label{sec:cvaml:bertsekas}

For convenience, we quote some results from~\textcite{bertsekasshreve1978}.
These are used in the proof of Lemma~\ref{lem:cvaml:deterministic_representation_lemma}.
While some of the definitions are rather technical, it is mostly sufficient to see a \emph{stochastic kernel} as the continuous generalization of the transition matrix in finite MDPs.
The projection used in \autoref{prop:cvaml:733} is simply a restriction of a set of tuples $(x,y)$ on the $x$ values.
Other topological statements are standard and can be found in textbooks such as~\textcite{munkres2018}.


\begin{proposition}[Proposition~7.30]
\label{prop:cvaml:730}
    Let $\mathcal{X}$ and $\mathcal{Y}$ be separable metrizable spaces and let $q(\mathrm{d}y|x)$ be a continuous stochastic kernel on $\mathcal{Y}$ given $\mathcal{X}$. If $f\in\mathcal{C}(\mathcal{X}\times \mathcal{Y})$, the function $\lambda: \mathcal{X} \rightarrow \mathbb{R}$ defined by
    \[
        \lambda(x) = \int f(x,y) q(\mathrm{d}y|x)
    \]
    is continuous.
\end{proposition}

\newcommand{\proj}{\text{proj}}
\begin{proposition}[Proposition~7.33]
\label{prop:cvaml:733}
    Let $\mathcal{X}$ be a metrizable space, $\mathcal{Y}$ a compact metrizable space, $\mathcal{D}$ a closed subset of $\mathcal{X}\times\mathcal{Y}$, $\mathcal{D}_x = \{y | (x,y) \in \mathcal{D} \}$, and let $f:\mathcal{D}\rightarrow \mathbb{R}^*$ be lower semi-continuous.
    Let $f^*:\proj_\mathcal{X}(\mathcal{D}) \rightarrow \mathbb{R}^*$ be given by \[
    f^*(x) = \min_{y \in \mathcal{D}_x} f(x,y).
    \]
    Then $\proj_\mathcal{X}(\mathcal{D})$ is closed in $\mathcal{X}$, $f^*$ is lower semi-continuous, and there exists a Borel-measurable function $\varphi: \proj_\mathcal{X}(\mathcal{D}) \rightarrow \mathcal{Y}$ such that $\text{range}(\varphi) \subset \mathcal{D}$ and \[
    f\squareb{x, \varphi(x)} = f^*(x), \quad \forall x \in \proj_\mathcal{X}(\mathcal{D}).
    \]
\end{proposition}

In our proof, we construct $f^*$ as the minimum of an IterVAML style loss and equate $\varphi$ with the function we call $f$ in our proof.
The change in notation is chosen to reflect the modern notation in \ac{mbrl} -- in the textbook, older notation is used.
\label{app:cvaml:conjecture}
