\chapter{Introduction}

Agents that act autonomously and intelligently in a complex world are in many ways the ultimate goal of AI research.
As we leave the constraints of well-defined environments and pre-specified behaviors, the ability to learn from trial and error becomes one of the most important challenges for such agents.
An industrial robot in a highly controlled environment, such as a car assembly line, can be programmed to behave reliably and safely by human engineers.
However, if we want robots to assist us in the real world outside of such controlled circumstances, we need to endow them with the ability to test out new behaviors and learn from their surroundings.
This goal has been pursued most comprehensively under the banner of \emph{reinforcement learning} \parencite{suttonbook}, in which an agent learns by sequentially interacting with an environment and receiving a scalar reward.

The core idea behind reinforcement learning is that an agent is able to learn from its own experiences and interactions with the world through receiving a reward which it seeks to optimize.
The agent acts in the world, obtains (or does not obtain) a reward and then seeks to repeat actions that lead to high reward and avoid actions which do not.
Over time, through trial and error, this learning scheme leads to reward-maximizing behavioral policies which an agent can execute to complete the task that is specified by the reward.

The reinforcement learning approach to artificial intelligence is a very general recipe and has received much attention over the years, with breakthrough successes in game playing \parencite{mnih2013playing,silver2016mastering}, robotics \parencite{smith2023demonstrating}, natural language processing and generation \parencite{ouyang2022training,rafailov2023direct,bi2024deepseek}, and technology \parencite{degrave2022magnetic}.
However, despite its broad applicability, reinforcement learning approaches are also often criticized for being unstable and hard to tune \parencite{irpan2018deep,henderson2018deep,patterson2024empirical}.
Many aspects leading to this oft-discussed instability have only recently been put into focus.
Recent work has identified culprits such as the failure of value functions to properly converge to good solutions \parencite{kumar2021implicit}, lack of generalization and observational overfitting \parencite{song2020observational,kirk2023survey}, and the diminishing ability of agents to learn consistently over many interaction steps \parencite{nikishin2022primacy,hussing2024dissecting}.
All of these reasons lead to a world in which, despite promising results, reinforcement learning has not achieved widespread adoption in many domains to which it is ostensibly well suited, such as household robotics.


\section{The problem of unstable value function learning}

Can we address the instability of reinforcement learning in practice?
To approach this question, we will focus primarily on the (in)stability of value function learning.
The value function is a central concept in reinforcement learning.
Intuitively, it summarizes the expected future return of executing a policy in a state.
Of course, we will establish a proper formal definition in the next chapter.
Value functions can be used to judge the impact of taking an action, to compare and improve policies, and to guide exploration.
Given their broad usefulness, a lot of effort has been devoted to obtain good value function learning algorithms, especially using deep neural networks \parencite{mnih2013playing,hasselt2010double,hasselt2016deep,fujimoto2018addressing,haarnoja2018sac}.
However, stable and efficient value learning in online scenarios with function approximation remains an important unsolved problem, and many recent works show persistent problems with existing learning algorithms \parencite{kumar2021implicit,nikishin2022primacy,hussing2024dissecting}.

It is important to note that there is no one single reason for the difficulty of value function learning.
As \textcite{suttonbook} famously states ``the danger of instability and divergence arises whenever we combine all of the following three elements, making up what we call the deadly triad: function approximation [...], bootstrapping [...], off-policy training''.
Many (deep) reinforcement learning algorithms combine all of these elements, and dealing with the deadly triad is therefore a common challenge that crops up in many works, explicitly or implicitly.

However, this does not mean that we cannot tackle the specific issues that contribute to the overall issue of instability, understand them and mitigate them to achieve strong improvements in our algorithms. 

\section{Learning to represent environments}

Our way to address instability will be to investigate the way information about the environment is represented so that it is useful for value estimation.
Over the course of the thesis we will see that this problem is strongly linked with many of the mechanisms underlying the deadly triad.

When building algorithms to act in unknown environments, one core challenge is to model the surroundings of the agent efficiently and correctly \parencite{vaml}.
The industrial robot can be tightly controlled and we can pre-specify what it will encounter in its limited environment and how to react to it.
But an autonomous agent needs to be able to perceive all of its surroundings and decide what information is necessary for its task.
In order to make decisions, such agents therefore need to filter and process their sensory input and learn about the structure of the environment, the consequences of their actions, and the rewards associated with different actions.

In supervised learning, neural networks have shown a remarkable ability to extract relevant information automatically from raw data, such as images or text, to solve complex tasks \parencite{goodfellow2016deep}.
However, in reinforcement learning, where an agent needs to learn how to act in an environment through trial and error, this task has proven to be more difficult \parencite{jaderberg2017reinforcement,igl2021transient,lyle2021effect,kumar2021implicit,nikishin2022primacy,hussing2024dissecting}.
As the agent explores the environment gradually, it is not easy to distinguish from the start what information is relevant for the decision making task and what is not \parencite{igl2021transient,voelcker2022value}.
Failing to do so often leads to agents getting stuck in suboptimal policies \parencite{kumar2021implicit}, overfitting to irrelevant aspects of the environment \parencite{song2020observational}, or failing to learn anything as they lack the computational capacity to model the world \parencite{kumar2021implicit,nikishin2022primacy}.

To understand how we can properly represent task-relevant information in reinforcement learning, we will take a close look at two concepts which turn out to be intimately related: implicit representations in the form of features learned by neural networks during training, and explicit representations in the form of world models which predict the behavior of the environment.
Neural network representation are learned functions which map the sensory input of an agent, such as a camera feed or low-level joint information of a robot, to a vector that is useful for reinforcement learning \parencite{ferns2004metrics,jaderberg2017reinforcement,abel2020thesis,le2021metrics}.

A world model allows an agent to predict how its actions impact the environment, and therefore it can be used to improve value estimation or for planning \parencite{dyna,mbpo,hafner2020dream,schrittwieser2020mastering}.
As we will see, learning world models provides an excellent way to obtain good feature representations as well.


\section{Tackling instability with decision-awareness and general-purpose learning}


To properly model task-relevant information, this thesis analyzes, and subsequently combines, \emph{decision-aware} algorithms, approaches that learn tailored representations and world models for the decision-making task and agent is attempting, with \emph{general-purpose} algorithms, approaches that simply attempt to generate representations and models which can be used for any downstream task.

\emph{Decision-awareness} describes the idea that the components of a learning approach for decision making, such as representations and models, should account for the final decision task \parencite{vaml,grimm2020value,abachi2020policy,nikishin2021control}.
With this understanding, we can also group most model-free value learning approaches such as DQN \parencite{mnih2013playing} or TD3 \parencite{fujimoto2018addressing} under the umbrella of decision-aware methods.

In many cases, representations and world models can also be trained in a \emph{general-purpose} way using objectives such as maximum likelihood estimation
However this can be inefficient or suboptimal, as computational resources are wasted on learning irrelevant aspects of the decision-making task \parencite{vaml}.
Instead, a decision-aware algorithms will account for the decision making task and e.g. put more emphasis on certain parts of the observation space, such as a traffic light, instead of others, such as clouds moving in the background of a video feed.

While this brief introduction seems to suggests that we should always use decision-aware approaches, experimental evidence complicates this.
Even though decision-aware algorithms promise to be more resource efficient, training with general-purpose methods, for example fitting a world model with a maximum likelihood objective, turns out to be more stable in practice \parencite{lovatto2020decision,voelcker2022value}.
Decision-aware algorithms need to continually adapt to the decision task as more information becomes available, while general purpose methods only have a single, stable training objective throughout the learning process.
Especially in the beginning of training when available information about the decision task is limited, this can lead to instability or suboptimal convergence of the learning algorithm.

Addressing this challenge is the goal of this thesis.
To this end, I propose the following hypothesis which will guide us to the rest of the work.

\begin{quote}
	To solve the persistent issues of \emph{instability} and \emph{inefficiency} of value function learning and to successfully leverage \emph{decision-aware} learning, an agent needs to effectively combine available information on the decision task with general-purpose learning mechanisms.
\end{quote}

Achieving this goal requires us to understand how we can build strong algorithms that use the best of both decision-aware and general purpose approaches.
We will discuss how such algorithms can be built, based on theoretical analysis and experimental evidence.

\section{Thesis structure}

Before diving into the solutions to the problem outlined above, \autoref{chap:background} presents a thorough overview of important definitions and algorithms in the field of reinforcement learning.
This serves as the background for all further discussions in the thesis.

\autoref{chap:overestimation} presents insights into how instability emerges in value function learning.
Experiments show that one of the major causes of unstable learning is diverging intermediate representations in the neural network architectures.
Based on work conducted in \textcite{hussing2024dissecting}, we can conclude that normalizing these representations has a vital stabilizing effect.

\autoref{chap:understanding} dives deeper into the question of stabilizing representations for value function learning.
While several different methods have been proposed in the literature, it remains an open question which model-based auxiliary tasks can be used as general-purpose methods to stabilize learning without deteriorating the quality of the representations for the decision making tasks.
This chapter presents both theoretical and empirical evidence that latent self-prediction models provide beneficial features for this goal.

\autoref{chap:vagram} establishes a second angle on learning stability and the tension between decision-awareness and general-purpose algorithms and focuses on model-based approaches.
Here we investigate how the instability of value function learning can harm the training of decision-aware environment models.
Empirical evidence is presented that suboptimal value function learning leads to a chicken-and-egg problem: without a good value function, we cannot train a good decision-aware model, but without a good decision-aware model, we cannot improve our value function.
To escape this cycle, we can modify a general purpose MSE loss with a sensitivity score derived from the current value function estimate.
This again highlights how combining both decision-aware and general purpose methods allows us to overcome the challenges of either.

\autoref{chap:cvaml} shows how previous attempts at value-aware model learning, including the IterVAML \parencite{itervaml} and MuZero \parencite{schrittwieser2020mastering} losses can be unified into one family. 
We then discuss issues that arise in this framework when using previously established losses naively for learning value functions with stochastic environment models, and show how to mitigate this problem.
The source of the problem is a subtle error term that appears when using common value-aware losses with stochastic models.
Furthermore this chapter discusses how latent model architectures, such as those in presented in \autoref{chap:understanding}, can be used to learn strong value-aware models.
With this, we fully unify the ideas of world model and representation learning.

\autoref{chap:mad} presents an empirically strong algorithm that combines the ideas presented in this thesis.
While previous chapters discuss the individual components of MAD-TD, here we put the focus back on solving the issue of instability established in \autoref{chap:overestimation}, tying all of our discussions together.
Using a latent value-aware model with well regularized features addresses a final important problem in value function learning: off-policy generalization.
With a learned model, counterfactual questions of the form ``what would have happened if the agent had taken action $a'$ in a past state $\state$'' can be answered, which empirically has a strong stabilizing effect on value function learning.

Finally, this thesis concludes with \autoref{chap:conclusion} where we review the findings and discuss remaining questions and promising directions for future work.


